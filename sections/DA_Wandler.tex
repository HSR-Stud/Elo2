\section{DA Wandler\hartl{455}}

\subsection{Parallelverfahren (Voltage Scaling)}
\begin{tabular}{|>{\bfseries}p{3cm}|c|p{6.6cm}|}
	\hline
	Strom-DAC \hartl{456} 
	& \includegraphics[width=7cm, valign=t]{pictures/Strom-DAC}
	& {\begin{align*}
		K &=2^N-1\\
		I &=\frac{V_{Ref}}{R}\\
		I_{Out} &=D \cdot I=D\cdot\frac{V_{Ref}}{R}\\
		I_{Out} &=D \cdot I\\
		I_{\bar{Out}} &=(K-D)\cdot I\\
		V_{Out} &=RI_{Out}-RI_{\bar{Out}} \\
			    &=(D-K)\cdot RI	
	  \end{align*}}
	  \begin{tabular}{lp{5cm}}
	  	K: & Anzahl Stromquellen \\
      	D: & Eingangswert (Anzahl Schalter die aktiv sind.)
      \end{tabular}
	\\ \hline
	String DAC \hartl{459}
	& \includegraphics[width=5.5cm, valign=t]{pictures/string_DAC}
	& \begin{description}
  		\item[Vorteile: ] garantierte Stetigkeit
  		\item[Nachteile:] benötigt $2^n$ Widerstände und $2^n$ Schalter, n-to-$2^n$ Decoder(linke Variante)
	  \end{description}
	  \[
	  	V_{Out}(D) = \frac{D}{2^n}(V_{Ref+} -V_{Ref-}) + V_{Ref-}
	  \]
	\\ \hline
	Segmented String DAC \hartl{459}
	& \includegraphics[width=5cm, valign=t]{pictures/segmented_string_DAC}
	& \begin{description}
  		\item[Vorteile: ] viel weniger Elemente
  		\item[Nachteile:] benötigt Buffer (offset-frei)
	  \end{description}
	\\ \hline
	Digitales Potentiometer \hartl{460}
	& \includegraphics[width=5cm, valign=t]{pictures/digitales_potentiometer}
	& \begin{description}
  		\item[Vorteile: ] automatisierter Elektronik-Test möglich
	  \end{description}
	\\ \hline
\end{tabular}


\subsection{Wägeverfahren\hartl{461}} 
\begin{tabular}{|>{\bfseries}p{4cm}|c|p{7.5cm}|}
	\hline
	Spannungs-summierung \hartl{461}
	& \includegraphics[width=5cm, valign=t]{./pictures/spannungssummierung.png}
	& {\begin{align*}
		V_{Out} &= \\
				&= \frac{B0 \cdot 2^0 \cdot G0+B1 \cdot 2^1 \cdot G0}{2^4 \cdot G0} \\
				& \quad \cdot \frac{B2 \cdot 2^2 \cdot G0+B3 \cdot 2^3 \cdot G0}{2^4 \cdot G0}\\
		        & \quad \cdot (V_{rep}-V_{refn})+V_{refn}
	  \end{align*}}
	  \begin{description}
  		\item[Vorteile:] N Widerstände, N Schalter
  		\item[Nachteile:] nicht garantiert stetig, grosse Wertebereiche für Widerstände, rechnen mit Leitwerten ($G_0 = \frac{1}{8R}$)
	  \end{description}
	\\ \hline
	Stromsummierung \hartl{462}
	& \includegraphics[width=5cm, valign=t]{./pictures/stromsummierung.png}
	& \begin{equation*}
		V_{Out}=(4+1) \cdot I
	  \end{equation*}
	\\ \hline
	Praktisch
	& \includegraphics[width=6cm, valign=t]{./pictures/praktisch.png}
	& \begin{equation*}
		Idac_{max}=\frac{V_{refp}-V_{refn}}{R} \cdot \frac{2^n-1}{2^n}
	  \end{equation*}
	\\ \hline
	R-2R-Netzwerk \hartl{462}
	& \includegraphics[width=6cm, valign=t]{./pictures/r2rnetzwerk.png}
	&
	\\ \hline
	Kapazitiver DAC
	& \includegraphics[width=6cm, valign=t]{./pictures/kapazitiverDAC.png}
	& {\begin{align*}
		C1	&=B3 \cdot 8C+B2 \cdot 4C+ \\
			& \quad +B1\cdot 2C+B0\cdot C \\
		C2	&= !B3 \cdot 8C+!B2 \cdot 4C+ \\
			& \quad +!B1\cdot 2C+!B0\cdot C+C \\
		V_{Out}& =\frac{C1}{C1+C2}\cdot (V_{refp}-V_{refn})\\
		\text{mit } & C1+C2=2^n\cdot C
	  \end{align*}}
	\\ \hline
\end{tabular}


\subsection{Zählverfahren(PWM)\hartl{466}}
\begin{tabular}{|>{\bfseries}p{4cm}|l|p{8cm}|}
	\hline 
	Grundprinzip \hartl{466}
	& \includegraphics[width=6cm, valign=t]{./pictures/pwm_DAC.png}
	& {\begin{gather*}
		V_{Out}=\frac{D}{2^n} \cdot (V_{refp}-V_{refn})+V_{refn}
	  \end{gather*}}
	  \begin{description}
  		\item[Vorteile:] einfache Schaltung, hohe Auflösung, Funktioniert ohne analoge Schaltung onchip (z.B. mit FPGA, Microcontroller) 
  		\item[Nachteile:] sehr langsam, benötigt grosse Zeitkonstante (Kapazität offchip)
	  \end{description}
	\\ \hline
	PWM-Ansteuerung \hartl{466}
	& \includegraphics[width=6cm, valign=t]{./pictures/pwm_Ansteuerung.png}
	& {\begin{equation*}
		\bar{V_{Out}}=\frac{n}{N}V_{Ref}
	  \end{equation*}}
	  \begin{tabular}{ll}
		N:&Takte\\
		n:&digitale Eingangsgrösse
	  \end{tabular}
	\\ \hline
\end{tabular}

\subsubsection{Kaskadierte DAC}
\begin{longtable}{|l|l|l|}
\hline
\begin{minipage}{4cm}
\textbf{Grundprinzip}
\end{minipage}
&
\begin{minipage}{6cm}
\includegraphics[width=6cm, height = 4cm]{pictures/kaskadiertDAC}
\end{minipage}
&

\begin{minipage}{8cm}
\begin{itemize}
  \item MS-DAC hat 2 Ausgangsspannungen (Über und unter dem gewünschten
  $V_{Out}$)
  \item LS-DAC hat kleine Eingangspannungsdifferenz $\to$ höhere Auflösung der
  Spannung
\end{itemize}
\end{minipage}
\\
\hline
\begin{minipage}{4cm}
\textbf{Zyklisch, algorithmischer DAC} \hartl{466}
\end{minipage}
&
\begin{minipage}{6cm}
\includegraphics[width=6cm, height = 3cm]{pictures/zyklischDAC}
\end{minipage}
&

\begin{minipage}{8cm}
\textbf{Ablauf der Wandlung}\\
\begin{enumerate}
  \item Die Spannung im S/H wird gelöscht (Schalter S1), S3 geöffnet)
  \item Schalter S1 wird danach auf den Verstärker-Ausgang geschaltet
  \item Laufvariale k wird auf 0 gesetzt
  \item Schalter S2 wird gesetzt: VREF oder GND ( abh. $D_{K}$).
  \item Der Addierer generiert sein Ausgangssignal
  \item Der Verstärker generiert sein Ausgangssignal
  \item Im S/H wird die Feedback-Spannung gespeichert (S1)
  \item X wird um 1 erhöht
  \item Gehe zu Schritt 4, wenn $X\leq n$
\end{enumerate}
\end{minipage}
\\
\hline
\begin{minipage}{4cm}
\textbf{Pipelined DAC}\\
\end{minipage}
&
\begin{minipage}{6cm}
\includegraphics[width=6cm, height = 2cm]{pictures/piplinedDAC}
\end{minipage}
&

\begin{minipage}{8cm}
Die Latenz beträgt n Zyklen, die Update-Frequenz ist aber n-mal grösser
\end{minipage}
\\
\hline
\begin{minipage}{4cm}
\textbf{Strom-DAC}
\end{minipage}
&
\begin{minipage}{6cm}
\includegraphics[width=6cm, height = 4cm]{pictures/stromDAC}
\end{minipage}
&

\begin{minipage}{8cm}
\begin{itemize}
  \item Stromspiegel
  \item MP0 ist gleich breit wie Stromquellen-MOS $\to$ I(MP0)=Iref
  \item MP1 ist doppelt so breit wie MP0 $\to I(MP1)=2*Iref$
  \item MP2 ist doppelt so breit wie MP1 $\to I(MP2)=4*Iref$
  \item \ldots
\end{itemize}
\end{minipage}
\\
\hline
\end{longtable}


\subsection{Multiplizierender DAC\hartl{470}} 

