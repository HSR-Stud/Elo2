\section{Referenzspannungen\hartl{276}} 

\subsection{Temperaturdrift von Referenzspannungen}
\begin{itemize}
  \item Wenn die Referenzspannung über den ganzen Temperaturbereich nicht mehr
  als 1 LSB driften darf:
\end{itemize}
\begin{figure}[!htbp]
\includegraphics[scale=0.5]{pictures/temperaturdrift}
\end{figure}
\subsection{Varianten von Referenzspannungen bei Wandlern}
\begin{figure}[!h]
\includegraphics[scale=0.3]{pictures/variantenReferenzspannungen}
\end{figure}


\subsection{Verschiedene Arten der Rerferenzspannungserzeugung\hartl{276}} 
\begin{longtable}{|l|l|l|}
\hline
\begin{minipage}{4cm}
\textbf{Einfachste "`Referenzspannungsquelle"'}
\end{minipage}
&
\begin{minipage}{6cm}
\includegraphics[width=6cm, height = 4cm]{pictures/spannungsteiler}
\end{minipage}
&
\begin{minipage}{8cm}
\begin{equation}
S_{V_{DD}}^{U_{Ref}}=\frac{\frac{\Delta
U_{Ref}}{U_{Ref}}}{\frac{\Delta V_{DD}}{V_{DD}}}=1
\end{equation}
\begin{itemize}
  \item S: Sensitivität
\end{itemize}
\end{minipage}
\\
\hline
\begin{minipage}{4cm}
\textbf{Dioden Referenz} \hartl{277}
\end{minipage}
&
\begin{minipage}{6cm}
\includegraphics[width=6cm, height = 4cm]{pictures/diodenReferenz}
\end{minipage}
&
\begin{minipage}{8cm}
\begin{gather}
U_{Ref}=U_{EB}=\frac{kT}{e}\ln{\frac{I}{I_{s}}} \notag\\\text{ für }V_{DD}>>
U_{EB}\\
S_{V_{DD}}^{U_{Ref}}=\frac{1}{\ln{\frac{I}{I_{S}}}}<1
\end{gather}
\end{minipage}
\\
\hline
\begin{minipage}{4cm}
\textbf{MOSFET Referenz}
\end{minipage}
&
\begin{minipage}{6cm}
\includegraphics[width=4cm, height = 4cm]{pictures/mosfetReferenz}
\end{minipage}
&
\begin{minipage}{8cm}
\begin{equation}
V_{REF}\approx \frac{R_{1}+R_{2}}{R_{2}}V_{GS}
\end{equation}
\end{minipage}
\\
\hline
\begin{minipage}{4cm}
\textbf{Bootstrap Referenz}
\end{minipage}
&
\begin{minipage}{6cm}
\includegraphics[width=6cm, height = 4cm]{pictures/bootstrapReferenz}
\end{minipage}
&
\begin{minipage}{8cm}
\begin{gather}
I_{R}=I_{D}\\
I_{R}R=U_{T}\ln{\frac{I_{D}}{I_{S}}}\\
U_{T}=\frac{kT}{q}
\end{gather}
\begin{itemize}
  \item Stromspiegel
  \item $U_{T}$: Temperaturabhängigkeit
\end{itemize}
\end{minipage}
\\
\hline
\begin{minipage}{4cm}
\textbf{Bandgap Referenz} \hartl{278}
\end{minipage}
&
\begin{minipage}{6cm}
\includegraphics[width=6cm, height = 4cm]{pictures/bandgapReferenz1}\\
\includegraphics[width=6cm, height = 4cm]{pictures/bandgapReferenz2}
\end{minipage}
&
\begin{minipage}{8cm}
\begin{gather}
I_{C}=B*I_{BS}(e^{\frac{V_{BE}}{V_{T}}}-1)\approx\notag\\\approx
B*I_{BS}*e^{\frac{V_{BE}}{V_{T}}}\\
\frac{I_{C1}}{I_{C2}}=e^{\frac{V_{BE1}-V_{BE2}}{V_{T}}} \to \Delta
V_{BE}=\notag\\=V_{BE1}-V_{BE2}=V_{T}\ln{\frac{I_{C1}}{I_{C1}}}\\
V_{T}=\frac{kT}{q}\\
V_{BE3}+V_{PTAT}\approx{!}1,2V
\end{gather}
\end{minipage}
\\
\hline
\begin{minipage}{4cm}
\textbf{Zenerdiode Referenz}
\end{minipage}
&
\begin{minipage}{6cm}
\includegraphics[width=6cm, height = 4cm]{pictures/zenerReferenz}
\end{minipage}
&
\\
\hline
\end{longtable}

\subsection{Zusammenfassung}
\begin{itemize}
  \item 2 Hauptaufgaben
  \begin{itemize}
    \item Spannung unabhängig von Speisung
    \item Spannung unabhängig von Temperatur
  \end{itemize}
  \item 2 Hauptprinzipien
  \begin{itemize}
    \item Zenerdioden
    \item Bandgap-Quellen
    \item Heute werden meistens integrierte Bandgap-Quellen verwendet, weil sie
    einfacher mit modernen Halbleiter-Technologien implementierbar sind.
    \end{itemize}
\end{itemize}
