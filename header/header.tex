%Schriftgr"osse, Layout, Papierformat, Art des Dokumentes
\documentclass[10pt,a4paper,fleqn,headsepline,footsepline]{scrartcl}
%Einstellungen der Seitenränder
\usepackage[left=0.5cm,right=0.5cm,top=0.3cm,bottom=0.3cm,includeheadfoot]{geometry}
% Sprache, Zeichensatz, packages
\usepackage[UTF8]{inputenc}
\usepackage[ngerman]{babel,varioref}
\usepackage{amssymb,amsmath,graphicx,xcolor,lastpage,wrapfig,verbatim}
\usepackage{tabularx,longtable}
\usepackage{array}
\usepackage{scrlayer-scrpage}
\usepackage{multirow,multicol}
\usepackage{trfsigns, trsym}
\usepackage{tikz}
\usetikzlibrary{fit}
\usepackage{circuitikz}
\usepackage{afterpage}

\usepackage{subcaption}
%\usepackage{subfigure}

\usepackage{enumitem}
\setlist{noitemsep,topsep=0pt,parsep=0pt,partopsep=0pt}
\usepackage{textcomp} % Für Grad zeichen

\usepackage{graphicx}

\usepackage{arydshln} % Für horizontale und vertikale punktierte Linien

\setlength{\mathindent}{0pt}
\raggedbottom


% Zum Bilder einfach in Tabellen einfügen (valign=t)
\usepackage[export]{adjustbox}
%
\setkomafont{pageheadfoot}{\footnotesize}
%
%
\RedeclareSectionCommands[
  beforeskip=-.2\baselineskip,
  afterskip=.1\baselineskip
]{section,subsection,subsubsection,paragraph}

\definecolor{pgrey}{rgb}{0.2,0.2,0.2}
\definecolor{black}{rgb}{0,0,0}
\definecolor{red}{rgb}{1,0,0}
\definecolor{white}{rgb}{1,1,1}
\definecolor{grey}{rgb}{0.8,0.8,0.8}
\definecolor{green}{rgb}{0,.8,0.05}
\definecolor{brown}{rgb}{0.603,0,0}

\DeclareMathOperator{\sinc}{sinc}
\DeclareMathOperator{\sgn}{sgn}
\DeclareMathOperator{\Real}{Re}
\DeclareMathOperator{\Imag}{Im}
%\DeclareMathOperator{\e}{e}
\DeclareMathOperator{\cov}{cov}
\DeclareMathOperator{\PolyGrad}{PolyGrad}

\newcommand{\HRule}{\noindent\rule{\linewidth}{1pt}}
%
\newcommand{\myparagraph}[1]{\paragraph{#1}\mbox{}\\\nopagebreak}
\newcommand{\formelbuch}[1]{$\quad{\textcolor{pgrey}{\mbox{\small{S#1}}}}$}
\newcommand{\hartl}[1]{$\quad{\textcolor{pgrey}{\mbox{\small{S#1}}}}$}


\newcommand*{\diff}{\mathop{}\!\mathrm{d}}
\newcommand{\FT}
{
\begin{picture}(1,0.5)
\put(0.2,0.1){\circle{0.14}}\put(0.27,0.1){\line(1,0){0.5}}\put(0.77,0.1){\circle*{0.14}}
\end{picture}
}

\newcommand{\IFT}
{
\begin{picture}(1,0.5)
\put(0.2,0.1){\circle*{0.14}}\put(0.27,0.1){\line(1,0){0.45}}\put(0.77,0.1){\circle{0.14}}
\end{picture}
}


\newcommand{\arraystretchOriginal}{1.5}
\renewcommand{\arraystretch}{\arraystretchOriginal}


\newcolumntype{L}[1]{>{\raggedright\let\newline\\\arraybackslash\hspace{0pt}}m{#1}}
\newcolumntype{C}[1]{>{\centering\let\newline\\\arraybackslash\hspace{0pt}}m{#1}}
\newcolumntype{R}[1]{>{\raggedleft\let\newline\\\arraybackslash\hspace{0pt}}m{#1}}


\author{\authorinfo}
\title{\titleinfo}
%
%Kopf- und Fusszeile
%
\lohead*{\titleinfo\quad \versioninfo}
\cohead*{\includegraphics[width=1.6cm]{header/small.png}}
\rohead*{\today}
\lofoot*{\authorinfo}
\cofoot*{}
\rofoot*{Seite \thepage { }von \pageref{LastPage}}
%
\pagestyle{scrheadings}

